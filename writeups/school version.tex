%
% Working notes for political gerrymandering problem.
%
\magnification=1095 
%\hsize=5.41in \vsize=7.5in     %for magnification=1200
 \hsize=5.91in \vsize=8.18in    %for magnification=1095
\parindent=20pt \parskip=4pt plus 4pt minus 4pt
\baselineskip=13pt plus 2pt minus 1pt \lineskiplimit=2pt
\lineskip=2pt plus 2pt \tolerance=800			%\raggedright

\font\eightmi=cmmi8 \font\eightsy=cmsy8
\font\rmeight=cmr8  \def\rmVIII{\rmeight \baselineskip=10pt
 plus 2pt minus 1pt \lineskiplimit=2pt \lineskip=2pt plus 1pt
 \parskip=0pt \textfont0=\rmeight \textfont1=\eightmi \textfont2=\eightsy
 \def\strut{\vrule height 7.5pt depth 2.5pt width 0pt}}
\font\caps=cmcsc10
%\font\bfVIII=cmbx8  \font\slVIII=cmsl8  \font\itVIII=cmti8 
%\font\capsVIII=cmcsc10 at 8pt  \font\hlv=phvr 
%\font\rmXVIII=cmr10 at 18truept
% Bold mathematical letters, e.g., $\bm\beta$.
%\font\tenbmi=cmmib10 \font\sevenbmi=cmmib7 \font\fivebmi=cmmib5
%\textfont11=\tenbmi \scriptfont11=\sevenbmi \scriptscriptfont11=\fivebmi
%\def\bm#1{{\fam11 #1}}

\input epsf % For Postscript figures: {\epsfxsize=  \epsffile{}}

% Color definitions.
%\input colordvi
%\def\BEGINC{\textPeach} \def\ENDC{\textBlack}
%\let\BEGINC=\relax \let\ENDC=\relax

%\newtoks\footlinesave \footlinesave=\footline % for restoring pagination
%\nopagenumbers

%\def\today{\ifcase\month\or January\or February\or March\or April\or
%May\or June\or July\or August\or September\or October\or November\or
%December\fi\ \number\day, \number\year}

%\newdimen\digitwidth \setbox0=\hbox{\rm0}  % For spacing in tables.
%\digitwidth=\wd0
%\catcode`?=\active \def?{\kern\digitwidth} % Inside table box.

% math symbol definitions
%\def\g{$\gamma$ }\def\ra{\rho_a}\def\ral{\ra(\lambda,x)}
%\def\Dral{D\ra(\lambda,x)}
%\def\tn#1{\left\Vert #1\right\Vert_2}
%\def\mn#1{\left\Vert #1\right\Vert_\infty}
%\def\Th#1\par{\par{\it\noindent{\caps Theorem}. #1}\par}
\def\lee{\mathrel{\vcenter{\hbox{$\scriptstyle\mathord<$}\nointerlineskip
 \vskip 1pt\hbox{$\scriptstyle\mathord=$}}}}
\def\gee{\mathrel{\vcenter{\hbox{$\scriptstyle\mathord>$}\nointerlineskip
 \vskip 1pt\hbox{$\scriptstyle\mathord=$}}}}
\def\real{\mathop{\rm I\!R}\nolimits}
\def\Rn{\real^n}
%\def\En{E^n}\def\Enn{E^{n\times n}}
\def\Rnn{{\bf R}^{n\times n}}
%\def\bull{\hfill \vrule height6pt width6pt depth0pt}
%\def\QED{\hfil\penalty 0\null\nobreak\hfill\quad\hbox{\rm Q. E. D.}\par}
%
% other definitions.
%\def\h{\par\hangindent=20pt\hangafter=1\noindent}
%
% for underlining multiple words: {\uline text }
%\def\uline#1 {\underbar{#1} \futurelet\next\ulinecheck}
%\def\ulinecheck{\ifcat\egroup\noexpand\next \unskip \else \uline\fi}
%
% TeX editing macros:  to underline, double-underline, wavy underline,
% horizontal strike out, slanted strike out "T", use
% \uline{T}, \uuline{T}, \uwave{T}, \sout{T}, \xout{T}, respectively.
%\input ulem.sty \input colordvi
%\def\BC#1#2{\sout{#1}{\textPeach #2\textBlack}} %TeX source markup.
%
% for highlighted blurbs:
%\def\blurb#1{\vadjust{\smallbreak\smallskip  % between horizontal lines
%\line{\vbox{\hrule \vskip 1 pt\hrule height 2 pt \hbox
%{\vbox{\smallskip\vbox{\spaceskip=3.3333pt plus 5pt minus 1.11111pt
%\xspaceskip=5.55555pt plus 5pt \hyphenpenalty=200\hsize=6.0truein
%{\parindent=10pt\narrower\noindent \hlv #1\par}}\smallskip}}
%\hrule \vskip 1 pt\hrule height 2pt
%}}\medskip}}
%\def\boxedblurb#1{\vadjust{\smallbreak\smallskip  % inside box
%\line{\hfil\hbox{\vrule\vbox{\hrule \vskip 1 pt\hbox
%{\kern 1pt\vrule width 2pt\vbox{\hrule height 2 pt\smallskip
%\vbox{\spaceskip=3.3333pt plus 5pt minus 1.11111pt
%\xspaceskip=5.55555pt plus 5pt \hyphenpenalty=200\hsize=5.5truein
%{\parindent=5pt\narrower\noindent \hlv #1\par}}\smallskip
%\hrule height 2 pt}\vrule width 2pt\kern 1pt}\vskip 1 pt\hrule
%}\vrule}\hfil}\medskip}}
%
% Set up reference labels. Usage: \rn\Name in prologue and \Name\ in
%				  text instead of [1].
%\newcount\refNUM \refNUM=0
%\def\rn#1{\advance\refNUM by 1
%	\edef#1{\hbox{[\number\refNUM]}}}

\def\u{{\vphantom g}_} \def\EnS{\u En\u S} \def\MnS{\u Mn\u S}
\def\HnS{\u Hn\u S}
Public school districts (typically at the city or county level) are
decomposed into {\it school attendance zones\/} (SAZs), which are clusters
of {\it student planning areas\/} (SPAs), typically at the neighborhood
or smaller level. Let the population data 
 $${\cal P} =\bigl\{\bigl(x,y,\u 0g,\u 1g,\ldots,\u {12}g,
 \u Eg, \u Mg,\u Hg, \u Ef,\u M\!f,\u H\!f\bigr)\bigr\},$$ where
$(x,y)$ are geographic coordinates of the approximate geographic center
of the student planning area (SPA), $\u 0g$, $\u 1g$, $\ldots$, $\u {12}g$
are the population sizes of grades 0(K), 1, $\ldots$, 12, respectively,
$\u Eg$, $\u Mg$, $\u Hg$ are the 5-year projected enrollments for elementary
(grades K--5), middle (grades 6--9), and high (grades 10--12) school
students, respectively, and $\u Ef$ $\u M\!f$, $\u H\!f$ are the numbers
of free and reduced meal students for elementary, middle, and high school,
respectively, for that SPA. The total number of SPAs is $n=|{\cal P}|$.
The school data for the elementary, middle, and high schools is ${\cal
S}_E$, ${\cal S}_M$, ${\cal S}_H$, respectively, where each ${\cal S}$ has
the form
 $$\bigl\{(x,y,c) \bigm| (x,y) \hbox{ are the geographic coordinates of
 the school, $c$ is the capacity}\bigr\}.$$
Let $\EnS=|{\cal S}_E|$, $\MnS=|{\cal S}_M|$, $\HnS=|{\cal S}_H|$
be the numbers of the different types of schools, and the data for school
$i$ of type $X$ will be denoted by $(\u Xx_i$, $\u Xy_i$, $\u Xc_i)$.

Assume that the student planning areas, which typically are geographically
small and compact with boundaries determined by the practical considerations
of bus routes, major streets, etc., are fixed.  To manage the requirement
of geographically connected school attendance zones (SAZs), let $\cal G$
be the ($n\times n$) adjacency matrix
 $${\cal G}_{ij} = \cases{1,& student planning areas $i$ and $j$ are deemed
 adjacent,\cr 0,& otherwise.\cr}$$
Here ``adjacent'' means having a common border of more than just a point.

Denoting student planning area $i$ by $\sigma_i = \bigl(x_i$, $y_i$, $\u
0g_i$, $\u 1g_i$, $\ldots$, $\u H\!f_i\bigr)$, an acceptable {\it school
districting\/} is three partitions $\u E{\cal Z}$, $\u M{\cal Z}$, $\u
H{\cal Z}$ of $\cal P$ of respective sizes $\EnS$, $\MnS$, $\HnS$
(the numbers of elementary, middle, and high schools, respectively), where
each subset (SAZ) of $\cal P$
 $$\u X{\cal Z}_i = \bigl\{\sigma_{j_1},\ldots,\sigma_{j_{\u Xm_i}}\bigr\},
 \quad i=1, \ldots,\u Xn_S,$$
($X$ above can be any of $E$, $M$, or $H$) in a partition must be
geographically connected, compact, and satisfy other physical, political,
and social constraints described later. School attendance zone
$\u X{\cal Z}_i$ is defined by its student planning area index set
$\u X{\cal I}_i = \bigl\{j_1$, $\ldots$, $j_{\u Xm_i}\bigr\}$.

The attendance at school $i$ of type $E$, $M$, $H$ is, respectively,
 $$\u E{\cal A}_i = \sum_{j\in\u E{\cal I}_i} \sum_{k=0}^5 \u kg_j, \quad
 \u M{\cal A}_i = \sum_{j\in\u M{\cal I}_i} \sum_{k=6}^9 \u kg_j, \quad
 \u H{\cal A}_i = \sum_{j\in\u H{\cal I}_i} \sum_{k=10}^{12} \u kg_j.$$
The physical capacity constraints of the schools are therefore
 $$\displaylines{(1-\tau)\u Ec_i \le \u E{\cal A}_i \le (1+\tau)\u Ec_i,
 \qquad i=1,\ldots,\EnS, \cr
 (1-\tau)\u Mc_i \le \u M{\cal A}_i \le (1+\tau)\u Mc_i,
 \qquad i=1,\ldots,\MnS, \cr
 (1-\tau)\u Hc_i \le \u H{\cal A}_i \le (1+\tau)\u Hc_i,
 \qquad i=1,\ldots,\HnS, \cr}$$
and the capacity constraints from the 5-year projected enrollment would be
 $$\displaylines{(1-\tau)\u Ec_i \le \sum_{j\in\u E{\cal I}_i} \u Eg_j
 \le (1+\tau)\u Ec_i, \qquad i=1,\ldots,\EnS, \cr
 (1-\tau)\u Mc_i \le \sum_{j\in\u M{\cal I}_i} \u Mg_j
 \le (1+\tau)\u Mc_i, \qquad i=1,\ldots,\MnS, \cr
 (1-\tau)\u Hc_i \le \sum_{j\in\u H{\cal I}_i} \u Hg_j
 \le (1+\tau)\u Hc_i, \qquad i=1,\ldots,\HnS, \cr}$$
where $\tau=0.2$ has been chosen by the School Board.

For nonempty $A,B \subset\{1,\ldots,n\}$, let ${\cal G}_{A,B}$ be the
submatrix of $\cal G$ with rows (columns) indexed by $A$ ($B$).  Thinking
of the graph defined by the student planning areas as nodes and arcs defined
by student planning area adjacency, each school attendance zone corresponds
to a subgraph. The school attendance zone is geographically connected if
its corresponding subgraph is connected. In terms of the adjacency matrix
$\cal G$, the constraint that $\u X{\cal Z}_i$ be geographically connected
is that the adjacency matrix ${\cal G}_{\u X{\cal I}_i, \u X{\cal I}_i}$
be irreducible. How to confirm this matrix property will be discussed in
detail later.

It is convenient to represent these partition subsets $\u X{\cal Z}_i$
and index sets $\u X{\cal I}_i$ by binary variables 
 $$\u X W_{ij} = \cases{1,& if SPA $j$ is assigned to SAZ $i$,\cr
 0,& otherwise,\cr}$$
for $i=1$, $\ldots$, $\EnS$ ($X=E$),
$i=1$, $\ldots$, $\MnS$ ($X=M$),
$i=1$, $\ldots$, $\HnS$ ($X=H$),
and $j=1$, $\ldots$, $n$.
Each of the 0-1 matrices $\u X W$
has exactly one 1 in each column. Define the population barycenter of
school attendance zone $\u X{\cal Z}_i$ for each school type $X$ as
 $$\displaylines{\u E(\bar x_i,\bar y_i) = \sum_{j\in\u E{\cal I}_i}
 \left({\sum\limits_{k=0}^5 \u kg_j\Big/ \u E{\cal A}_i}\right) (x_j,y_j),\cr
 \u M(\bar x_i,\bar y_i) = \sum_{j\in\u M{\cal I}_i}
 \left({\sum\limits_{k=6}^9 \u kg_j\Big/ \u M{\cal A}_i}\right) (x_j,y_j),\cr
 \u H(\bar x_i,\bar y_i) = \sum_{j\in\u H{\cal I}_i}
 \left({\sum\limits_{k=10}^{12} \u kg_j\Big/ \u H{\cal A}_i}\right)
 (x_j,y_j).\cr}$$
Achieving optimal school attendance zone compactness is then similar to
$K$-means clustering of the SPA centers using the 2-norm, and achieving
student proximity to schools uses the 1-norm (which captures travel time
better than the Euclidean 2-norm distance): assign student planning areas
$\sigma_j$ to $\EnS$, $\MnS$, $\HnS$ school attendance zones
$\u E{\cal Z}_i$, $\u M{\cal Z}_i$, $\u H{\cal Z}_i$, respectively, to
minimize 
 $$\eqalignno{
 \Phi(\u EW,\u M\!W,\u H\!W) = &\sum_{i=1}^{\EnS} \sum_{j\in\u E{\cal I}_i}
 \bigl\|(x_j,y_j) - \u E(\bar x_i,\bar y_i)\bigr\|_2^2 +
 \sum_{i=1}^{\MnS} \sum_{j\in\u M{\cal I}_i}
 \bigl\|(x_j,y_j) - \u M(\bar x_i,\bar y_i)\bigr\|_2^2 +{}\cr
 &\sum_{i=1}^{\HnS} \sum_{j\in\u H{\cal I}_i}
 \bigl\|(x_j,y_j) - \u H(\bar x_i,\bar y_i)\bigr\|_2^2 +
 \gamma\left( \sum_{i=1}^{\EnS} 
 \bigl\|(\u Ex_i,\u Ey_i) - \u E(\bar x_i,\bar y_i)\bigr\|_1 \right. +{}\cr
 &\left.\sum_{i=1}^{\MnS}
 \bigl\|(\u Mx_i,\u My_i) - \u M(\bar x_i,\bar y_i)\bigr\|_1 +
 \sum_{i=1}^{\HnS}
 \bigl\|(\u Hx_i,\u Hy_i) - \u H(\bar x_i,\bar y_i)\bigr\|_1 \right),\cr}$$
subject to the school capacity and school attendance zone
connectivity constraints. $\gamma\gg1$ weights the relative importance of
proximity to compactness. There are several alternative ways to deal with the
compactness and proximity objectives. For proximity, one could use all the SPA
center to school distances (within a SAZ) rather than the single SAZ
barycenter to school distance as above. One could also formulate a
multiobjective optimization problem with a compactness objective (first three
terms of $\Phi$) and a proximity objective (last three terms of $\Phi$),
however, each Pareto optimal solution of this multiobjective formulation
corresponds to some choice of $\gamma$ in the above formulation.

The binary variables $\u XW_{ij}$ simplify the expressions and calculations.
For instance, if a population vector $\u EP$ of length $n$ were formed with
$j$th element being $\sum_{k=0}^5 \u kg_j$, then the capacity constraint
 $$(1-\tau)\u Ec_i \le \u E{\cal A}_i \le (1+\tau)\u Ec_i$$
would become simply
 $$(1-\tau)\u Ec_i \le (\u EW \u EP)_i \le (1+\tau)\u Ec_i$$
in terms of $\u EW$.

Let $e=(1$, $\ldots$, $1)\in\Rn$, and $\hbox{diag}
({\cal G}e)$ be the $n\times n$ diagonal matrix whose diagonal elements are
the degrees of the nodes in the graph $G$ with adjacency matrix $\cal G$.
The {\it graph Laplacian\/} of the graph $G$ is defined as
 $$\Lambda(G) = \hbox{diag}({\cal G}e) - {\cal G},$$
and $G$ is connected if and only if $\cal G$ is irreducible if and only if
the second smallest eigenvalue of $\Lambda(G)$ is not zero. Hence the SAZ
$\u X{\cal Z}_i$ is connected if and only if the second smallest eigenvalue of
the graph Laplacian
 $$\hbox{diag}\bigl({\cal G}_{\u X{\cal I}_i,\u X{\cal I}_i} \>e\bigr) -
 {\cal G}_{\u X{\cal I}_i,\u X{\cal I}_i}$$
of the subgraph of $G$ corresponding to $\u X{\cal Z}_i$ is not zero;
$e\in\real^{\u Xm_i}$ here. Since the graph Laplacian matrix is symmetric and
positive semidefinite, the connectedness constraints for the SAZs can
be taken as
 $$0.001 - \bigl(\hbox{second smallest eigenvalue of diag}
 \bigl({\cal G}_{\u X{\cal I}_i,\u X{\cal I}_i} \>e\bigr) -
 {\cal G}_{\u X{\cal I}_i,\u X{\cal I}_i}\bigr) \le 0$$
for $i=1,\ldots,\u Xn_S$. Here, as always, $X$ represents any of $E$, $M$, or
$H$. An alternative to these eigenvalue constraints,
which cannot be expressed in closed form but are computationally cheap,
is to express graph connectivity by (exponentially many) linear constraints.
For brevity, denote the nodes of the graph $G$ by the student planning
area indices, and for $\emptyset \ne S \subset \{1$, $\ldots$, $n\}$, define the
neighborhood of $S$ by 
 $$N(S)=S \cup \bigl\{m\mid m \hbox{ is adjacent to some }j\in S\bigr\},$$
i.e., $N(S)$ represents all student planning areas either represented by $S$ or adjacent to some student planning area represented by $S$. 
Then the school attendance zones $\u X{\cal Z}_i$ are connected
if $\forall i=1$, $\ldots$, $\u Xn_S$, $\forall\ell=1$, $\ldots$, $n$,
$\forall S \subset\{1$, $\ldots$, $n\} \backslash N(\{\ell\}) \ne\emptyset$,
 $$\sum_{j\in N(S)\backslash S} \u XW_{ij} -
 \sum_{j\in S\cup\{\ell\}} \u XW_{ij} \ge 1-(|S|+1),$$
which is ${\cal O}(\u Xn_S n2^n)$ constraints.

The problem is clearly separable in the variables $\u EW$, $\u MW$,
$\u HW$, since the objective function $\Phi$ is separable, and the school
districting constraints for elementary, middle, and high schools are
completely independent of each other. While formulated here as a single
optimization problem, in practice the problems for the three school types
are solved separately. Having said this, the full problem is now stated.

For convenience, define population vectors $\u EP$, $\u MP$, $\u HP$ of
length $n$ with $j$th element being $\sum_{k=0}^5 \u kg_j$,
$\sum_{k=6}^9 \u kg_j$, $\sum_{k=10}^{12} \u kg_j$, respectively.
The variables are $\u EW \in\{0,1\}^{\EnS\times n}$,
$\u MW \in\{0,1\}^{\MnS\times n}$, and $\u HW \in\{0,1\}^{\HnS\times n}$.
The optimization problem is
 $$\displaylines{
 \min_{\u EW,\u MW,\u HW} \Phi\bigl(\u EW,\u MW,\u HW\bigr)\cr
 \hbox{subject to}\hfill\cr
 (1-\tau)\u Xc_i \le (\u XW \u XP)_i \le (1+\tau)\u Xc_i,
 \qquad \forall X\in\{E,M,H\} \quad\forall i,\cr
 (1-\tau)\u Xc_i \le \sum_{j=1}^n\u XW_{ij}\>\u Xg_j \le (1+\tau)\u Xc_i,
 \qquad \forall X\in\{E,M,H\} \quad\forall i,\cr
 0.001 - \bigl(\hbox{second smallest eigenvalue of diag}
 \bigl({\cal G}_{\u X{\cal I}_i,\u X{\cal I}_i} \>e\bigr) -
 {\cal G}_{\u X{\cal I}_i,\u X{\cal I}_i}\bigr) \le 0,\quad
 \forall X\in\{E,M,H\} \;\forall i.\cr}$$

\bye %%%%%%%%%


A typical approach (called scalarization) to solving multiobjective
optimization problems is to solve the problem
 $$\displaylines{\min_{Z\in\{0,1\}^{k\times n}}
 (1-\lambda)\Phi(Z) + \lambda\Psi(Z) \quad \hbox{subject to}\cr
 \sum_{i=1}^k Z_{ij}=1 \quad\hbox{for }j=1,\ldots,n,\cr
 \max_{1\le i<j \le k} |(ZP)_i-(ZP)_j| - {\cal S}_{\max} \le0,\cr
 0.001 - \bigl(\hbox{second smallest eigenvalue of diag}
 \bigl({\cal G}_{{\cal I}_i{\cal I}_i}e\bigr) -
 {\cal G}_{{\cal I}_i{\cal I}_i}\bigr) \le 0, \quad
 \hbox{for }i=1,\ldots,k,\cr}$$
for $\lambda$ in $0\le\lambda\le1$. The set of points
$\bigl(\Phi(\hat Z),\Psi(\hat Z)\bigr)$ in objective function space
corresponding to Pareto optimal solutions $\hat Z$ is called the {\it
Pareto front\/} or Pareto manifold, and (in the case here of two objectives)
may be a smooth convex or concave curve, or even a disconnected point set.
A practical difficulty is that the spacing of points along the Pareto front
(curve) may not correspond to the spacing between the corresponding $\lambda$
values.  A strategy for choosing $\lambda$ values to efficiently approximate
the Pareto front was addressed recently in [Deshpande et al., 2016].



\bye

